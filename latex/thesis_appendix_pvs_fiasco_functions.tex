\hypertarget{pvs_fiasco_functions}{\chapter{PVS: fiasco\_functions.pvs}}
\lstset{language=PVS}
\begin{lstlisting}
% -----------------------------------------------------------------------------
% The Fiasco IPC implementation is basically a set of inter-related functions.
% Here we define our PVS representation of those functions. Although most 
% functions can be mapped almost one-on-one to their C++ counterparts, some
% functions are actually simplified versions, which describe the behaviour of
% its Fiasco implementations at a more abstract level. There have also been 
% functions added that are not in the C++ source, in all cases this has been 
% done to split up larger functions into more compact parts, the functionality 
% of course remains intact.
% -----------------------------------------------------------------------------
fiasco_functions : THEORY
EXPORTING ALL WITH ALL
BEGIN
  % ---------------------------------------------------------------------------
  % Import the theories which will be used in our lemmas.
  % ---------------------------------------------------------------------------
  IMPORTING fiasco_types
  IMPORTING fiasco_states

  % ---------------------------------------------------------------------------
  % Functional specifications which will describe IPC in Fiasco.
  % ---------------------------------------------------------------------------  

  % Add bits to the state of a thread, all bits set in the <bits> thread state
  % will be set in the state of the thread pointed to by <tp>.
  state_add_dirty(tp: Thread_pointer, bits: Thread_state, 
                  state_old: System_state): System_state =
    LET
      % Determine which state to work with.
      state = state_old`threads(tp)`state,
      state =  state WITH 
        [
         thread_ready     := state`thread_ready     OR bits`thread_ready,
         thread_cancel    := state`thread_cancel    OR bits`thread_cancel,
         thread_dead      := state`thread_dead      OR bits`thread_dead,
         thread_busy      := state`thread_busy      OR bits`thread_busy,
         thread_invalid   := state`thread_invalid   OR bits`thread_invalid,
         thread_polling   := state`thread_polling   OR bits`thread_polling,
         thread_receiving := state`thread_receiving OR bits`thread_receiving,
         thread_ipc_in_progress      := state`thread_ipc_in_progress OR
                                        bits`thread_ipc_in_progress,
         thread_send_in_progress     := state`thread_send_in_progress OR
                                        bits`thread_send_in_progress,
         thread_transfer_in_progress := state`thread_transfer_in_progress OR
                                      bits`thread_transfer_in_progress
        ]
    IN
      state_old WITH [(threads)(tp)`state := state]

  % Delete bits from the state of a thread, all bits set in the <bits> thread 
  % state will be unset in the state of the thread pointed to by <tp>.
  state_del_dirty(tp: Thread_pointer, bits: Thread_state, 
                  state_old: System_state): System_state =
    LET
      % Determine which state to work with.
      state = state_old`threads(tp)`state,
      state = state WITH 
      [
       thread_ready     := state`thread_ready     AND NOT bits`thread_ready,
       thread_cancel    := state`thread_cancel    AND NOT bits`thread_cancel,
       thread_dead      := state`thread_dead      AND NOT bits`thread_dead,
       thread_busy      := state`thread_busy      AND NOT bits`thread_busy,
       thread_invalid   := state`thread_invalid   AND NOT bits`thread_invalid,
       thread_polling   := state`thread_polling   AND NOT bits`thread_polling,
       thread_receiving := state`thread_receiving AND NOT bits`thread_receiving,
       thread_ipc_in_progress      := state`thread_ipc_in_progress AND NOT
                                      bits`thread_ipc_in_progress,
       thread_send_in_progress     := state`thread_send_in_progress AND NOT
                                      bits`thread_send_in_progress,
       thread_transfer_in_progress := state`thread_transfer_in_progress AND NOT
                                      bits`thread_transfer_in_progress
      ]
    IN
      state_old WITH [(threads)(tp)`state := state]

  % Delete- and add bits to the state of the thread pointed to by <tp>; all bits
  % not set in the <mask> thread state will first be unset and then all bits
  % set in <bits> will be set.
  state_change_dirty(tp: Thread_pointer, mask: Thread_state,
                     bits: Thread_state,state_old: System_state): System_state =
    LET
      % Determine which state to work with.
      state = state_old`threads(tp)`state,
      state = state WITH 
          [
           thread_ready     := state`thread_ready     AND mask`thread_ready,
           thread_cancel    := state`thread_cancel    AND mask`thread_cancel,
           thread_dead      := state`thread_dead      AND mask`thread_dead,
           thread_busy      := state`thread_busy      AND mask`thread_busy,
           thread_invalid   := state`thread_invalid   AND mask`thread_invalid,
           thread_polling   := state`thread_polling   AND mask`thread_polling,
           thread_receiving := state`thread_receiving AND mask`thread_receiving,
           thread_ipc_in_progress      := state`thread_ipc_in_progress AND
                                          mask`thread_ipc_in_progress,
           thread_send_in_progress     := state`thread_send_in_progress AND
                                          mask`thread_send_in_progress,
           thread_transfer_in_progress := state`thread_transfer_in_progress AND
                                          mask`thread_transfer_in_progress
          ],

      state = state WITH 
        [
         thread_ready     := state`thread_ready     OR bits`thread_ready,
         thread_cancel    := state`thread_cancel    OR bits`thread_cancel,
         thread_dead      := state`thread_dead      OR bits`thread_dead,
         thread_busy      := state`thread_busy      OR bits`thread_busy,
         thread_invalid   := state`thread_invalid   OR bits`thread_invalid,
         thread_polling   := state`thread_polling   OR bits`thread_polling,
         thread_receiving := state`thread_receiving OR bits`thread_receiving,
         thread_ipc_in_progress      := state`thread_ipc_in_progress OR
                                        bits`thread_ipc_in_progress,
         thread_send_in_progress     := state`thread_send_in_progress OR
                                        bits`thread_send_in_progress,
         thread_transfer_in_progress := state`thread_transfer_in_progress OR
                                        bits`thread_transfer_in_progress
        ]
    IN
      state_old WITH [(threads)(tp)`state := state]

  % Delete bits from the state of a thread, all bits set in the <bits> thread 
  % state will be unset in the state of the thread pointed to by <tp>.
  %
  % NOTE: in our model this function is just an interface for the 
  % state_del_dirty() function. However, in Fiasco there is a real difference
  % between the two functions as the dirty version assumes that the CPU lock
  % is held, which allows for faster updating. As we have not modelled the CPU
  % lock, the two functions are the same in our model. We have chosen to still
  % include the state_del() function because it allows for a better reflection
  % of the source code and the model might later be changed to allow CPU locks.
  state_del(tp: Thread_pointer, bits: Thread_state, 
            state_old: System_state): System_state =
    state_del_dirty(tp,bits,state_old)

  % Delete- and add bits to the state of the thread pointed to by <tp>; all bits
  % not set in the <mask> thread state will first be unset and then all bits
  % set in <bits> will be set.
  %
  % NOTE: here we have a similar situation to the one described in the notes of
  % the state_del() function, please check its notes above for more information.
  state_change(tp: Thread_pointer, mask: Thread_state, bits: Thread_state, 
               state_old: System_state): System_state = 
    state_change_dirty(tp,mask,bits,state_old)
  
  % Remove the thread lock on the partner thread by setting the thread lock
  % owner to the zero thread.
  clear_dirty(partner: Thread_pointer, state_old: System_state): System_state =
    state_old WITH [(threads)(partner)`thread_lock := Zero_thread]

  % Remove the thread lock on the partner thread by setting the thread lock
  % owner to the zero thread.
  %
  % NOTE: although this specification is equal to that of clear_dirty(), once
  % again this is due to our model being a simplified version of the source 
  % code. In the source code, the clear_dirty() function possibly switches the
  % execution context, whereas the clear_dirty_dont_switch() function obviously
  % does not. As we do not model execution context switches, the result of the
  % two functions is the same. We have still included both functions because of 
  % the reasons also specified in the notes of the state_del() function.
  clear_dirty_dont_switch(partner: Thread_pointer, 
                          state_old: System_state): System_state =
    state_old WITH [(threads)(partner)`thread_lock := Zero_thread]

  % Set the thread lock of the partner thread to the <this> thread.
  %
  % NOTE: this is a typical function in which a lot of implementation details
  % have been abstracted away. In the C++ implementation, the function first
  % checks if another thread has locked the partner and if so, it enters a loop
  % that waits for the lock to be released. The end result is always that the
  % lock is acquired by the <this> thread so this lead to our simplified version.
  lock_dirty(partner: Thread_pointer,state_old: System_state): System_state =
    state_old WITH [(threads)(partner)`thread_lock := state_old`this]

  % Indicates if a partner is ready for a sender.
  sender_ok(sender: Thread_pointer, receiver: Thread_pointer,
            state_old: System_state): bool =
    % The receiver should at least be in a receiving IPC state.
    IF NOT state_old`threads(receiver)`state`thread_receiving OR
       NOT state_old`threads(receiver)`state`thread_ipc_in_progress
    THEN
      false
    % Open wait: no partner should be specified and the sender should be the 
    % first in the receiver's sender list.
    ELSIF state_old`threads(receiver)`partner = Zero_thread AND
          state_old`threads(receiver)`sender_list = First THEN 
      true
    % Closed wait: check if the sender is the same as the partner specified.
    ELSIF sender = state_old`threads(receiver)`partner THEN
      true
    ELSE
      false
    ENDIF

  % Set the partner of the receiver to the specified sender.
  set_partner(sender: Thread_pointer, receiver: Thread_pointer,
              state_old: System_state): System_state =
    state_old WITH [(threads)(receiver)`partner := sender]
  
  % Initialize IPC between the sender and receiver.
  ipc_init(sender: Thread_pointer, receiver: Thread_pointer,
           state_old: System_state): System_state =
    % Set the <thread_transfer_in_progress> bit on the receiver state.
    LET 
      % Set the partner.
      state_temp = set_partner(sender,receiver,state_old),
      bits = TS_empty WITH [thread_transfer_in_progress := true]
    IN
      state_add_dirty(receiver,bits,state_temp)

  % Transfer a message between the sender and receiver.
  %
  % NOTE: this function completely ignores the actual copying of the message,
  % we assume that the copying will always succeed. What's more, this function 
  % does not do anything at all because we totally abstracted away the sending
  % of the message.
  transfer_msg(receiver: Thread_pointer,state_old: System_state): System_state =
    state_old

  % Indicates if the receiver is engaged in IPC with the sender.
  in_ipc(sender: Thread_pointer, receiver: Thread_pointer, 
         state_old: System_state): bool =
    state_old`threads(receiver)`state`thread_transfer_in_progress AND
    state_old`threads(receiver)`state`thread_ipc_in_progress AND NOT
    state_old`threads(receiver)`state`thread_cancel AND 
    state_old`threads(receiver)`partner = sender

  % Wakeup the receiver, which ensures that the state is changed to reflect that
  % it is no longer involved in IPC and is ready to continue with other tasks. 
  wake_receiver(receiver:Thread_pointer, state_old:System_state): System_state =
    % Define the mask and bits to change the state of the receiver.
    LET mask = TS_full WITH [thread_busy := false,
                             thread_receiving := false,
                             thread_transfer_in_progress := false,
                             thread_ipc_in_progress := false],
        bits = TS_empty WITH [thread_ready := true]
    IN
      state_change_dirty(receiver,mask,bits,state_old)

  % Return a (random) sender list status.
  %
  % NOTE: the function is defined without an implementation, that means that we 
  % cannot predict the return value of the function. However, as it still is a
  % function that means that equal input parameters result in the same return
  % values. Therefore, to really achieve the randomness we are looking for we
  % need to make sure that the input parameter never has the same value twice.
  sender_list_status(n: nat): Sender_list_status

  % Enqueue the <this> thread in the receiver's sender list.
  sender_enqueue(receiver:Thread_pointer,state_old:System_state): System_state =
    LET 
      % Update the seed and get a random sender list status.
      state_old = state_old WITH [seed := state_old`seed + 1],
      status = sender_list_status(state_old`seed)
    IN
    % Check if the sender list is empty, if so the <this> thread will become
    % the first item in the sender list and otherwise it will be set to an
    % enqueued status (indicating that it is in the sender list but is not
    % the first item).
    IF status = Empty THEN
      state_old WITH [(threads)(receiver)`sender_list := First]
    ELSE
      state_old WITH [(threads)(receiver)`sender_list := Enqueued]
    ENDIF

  % Dequeue the <this> thread from the receiver's sender list.
  sender_dequeue(receiver:Thread_pointer,state_old:System_state): System_state =
    state_old WITH [(threads)(receiver)`sender_list := Dequeued]

  % Dequeue the <this> thread from the receiver's sender list.
  sender_dequeue_head(receiver: Thread_pointer, 
                      state_old: System_state): System_state =
    state_old WITH [(threads)(receiver)`sender_list := Dequeued]

  % Indicates if the sender is in the receiver's sender list.
  in_sender_list(receiver: Thread_pointer, state_old: System_state): bool =
    NOT state_old`threads(receiver)`sender_list = Dequeued 

  % Prepare a thread for receiving a message from the specified sender.
  prepare_receive_dirty(sender: Thread_pointer, receiver: Thread_pointer, 
                        state_old: System_state): System_state =
    LET
      % Set the partner.
      state_temp = set_partner(sender,receiver,state_old),

      mask = TS_full WITH [thread_send_in_progress := false,
                           thread_polling := false,
                           thread_transfer_in_progress := false],
      bits = TS_empty WITH [thread_receiving := true, 
                            thread_ipc_in_progress := true]
    IN
      state_change_dirty(receiver,mask,bits,state_temp)
       
  % Kill the partner thread.
  kill(partner: Thread_pointer, state_old: System_state): System_state =
    % Both locked- and invalid threads cannot be killed
    IF NOT state_old`threads(partner)`thread_lock = state_old`this AND 
       NOT state_old`threads(partner)`state`thread_invalid
    THEN
      state_old WITH[(threads)(partner)`state`thread_dead := true]
    ELSE
      state_old
    ENDIF

  % Update the current thread's registers, which can be used to create a new
  % thread out of an existing one. This results in the <thread_cancel> bit
  % of the <this> thread being set, signifying that ongoing IPC should be
  % cancelled.
  sys_thread_ex_regs(tp: Thread_pointer,state_old: System_state): System_state =
    state_old WITH [(threads)(tp)`state`thread_cancel := true]
    
  % An IPC timeout has occured, which means that the specified timeout time has
  % passed. We simulate this by setting the <timeout> member to true.
  timeout(tp: Thread_pointer, state_old: System_state): System_state =
   LET
     % Define the state changed by the timeout.
     mask = TS_full WITH [thread_ipc_in_progress := false],
     bits = TS_empty WITH [thread_ready := true]
   IN
     state_change(tp,mask,bits,state_old) WITH [timeout := true]

  % Called by the receiver on a sender to determine if both are ready to engage
  % in IPC.  
  ipc_receiver_ready(sender: Thread_pointer, receiver: Thread_pointer,
                     state_old: System_state): bool =
    state_old`threads(sender)`state`thread_ipc_in_progress AND 
    sender_ok(sender,receiver,state_old)

  % In this function the side-effects of the ipc_receiver_ready() function are
  % handled.
  %
  % NOTE: in the Fiasco source code the receiver calls the ipc_receiver_ready()
  % function, which returns a boolean value indicating if the sender was ready
  % for the receiver. However, this function also has side-effects if the 
  % sender was indeed ready so we had to split up the function in one in which
  % a boolean value was returned and one in which the side-effects are handled.
  ipc_receiver_ready_change(sender: Thread_pointer, receiver: Thread_pointer,
                            state_old: System_state): System_state =
    LET
      % Initialize IPC.
      state_temp = ipc_init(sender,receiver,state_old),

      % Set the sender to ready and indicate that the transfer is in progress.
      bits = TS_empty WITH [thread_ready := true,
                            thread_transfer_in_progress := true],
      state_temp = state_add_dirty(sender,bits,state_temp),

      % Remove the ready state from the receiver.
      bits = TS_empty WITH [thread_ready := true]
    IN
      state_del_dirty(receiver,bits,state_temp)
   
  % This preemption action signifies a receiver becoming ready to receive a 
  % message from a sender.
  receiver_ready(sender: Thread_pointer, receiver: Thread_pointer,
                 state_old: System_state): System_state = 
    LET
      state_old = 
        % Check if the receiver has been initialized.
        IF NOT state_old`receiver_initialized THEN 
          % Prepare the receiver using a 'random' partner.
          prepare_receive_dirty(Receiver_partner_thread,receiver,state_old)
            WITH [(receiver_initialized) := true]
        ELSE
          state_old
        ENDIF
    IN
    % Make sure the receiver is in the right state to receive from a sender.
    IF state_old`threads(receiver)`state`thread_receiving AND
       state_old`threads(receiver)`state`thread_ipc_in_progress AND NOT
       state_old`threads(receiver)`state`thread_cancel
    THEN
      % Closed wait: check if the sender is the same as the partner specified.
      IF sender = state_old`threads(receiver)`partner THEN
        % Check if the sender meets the conditions of the receiver
        IF in_sender_list(receiver,state_old) AND 
           ipc_receiver_ready(sender,receiver,state_old)
        THEN
          % Apply the changes caused by the side-effects of the 
          % ipc_receiver_ready() function.
          ipc_receiver_ready_change(sender,receiver,state_old)
        ELSE
          state_old
        ENDIF
      % Open wait: no partner should be specified and the sender should be 
      % the first item.
      ELSE
        % Check if the sender is not ready in which case it should be removed
        % from the sender list.
        IF NOT ipc_receiver_ready(sender,receiver,state_old) THEN
          sender_dequeue_head(receiver,state_old)
        ELSE
          % Apply the changes caused by the side-effects of the
          % ipc_receiver_ready() function.
          ipc_receiver_ready_change(sender,receiver,state_old)
        ENDIF
      ENDIF    
    ELSE
      state_old
    ENDIF

  % Return a (random) preemption action.
  %
  % NOTE: the randomness achieved in the same way as described in the notes of
  % the sender_list_status() function.
  preemption_action(n: nat): Preemption_actions

  % Execute a list of preemption point actions.
  preemption_point_actions(actions: Preemption_actions, partner: Thread_pointer,
                           allow_timeout: bool, 
                           state_old: System_state): RECURSIVE System_state =
    CASES actions OF
      null: state_old,
      cons(action,remaining_actions):
      LET state_temp = 
        % Check if a call was made to sys_thread_ex_regs().
        IF action = Sys_thread_ex_regs THEN
          sys_thread_ex_regs(state_old`this,state_old)
        % Only allow a timeout to occur when specified, most of the time a 
        % timeout cannot occur.
        ELSIF action = Timeout AND allow_timeout THEN
          timeout(state_old`this,state_old)
        % Check if the partner is killed.
        ELSIF action = Kill THEN
          kill(partner,state_old)
        % A receiver can only become ready when it is not equal to the sender,
        % if so it would be waiting forever.
        ELSIF action = Receiver_ready AND NOT state_old`this = partner THEN
          receiver_ready(state_old`this,partner,state_old)
        ELSE
          state_old
        ENDIF
      IN
      % Execute the remaining actions.
      preemption_point_actions(remaining_actions,partner,allow_timeout,state_temp)
    ENDCASES
    MEASURE length(actions)

  % Simulate a preemption point by ignoring the actual scheduling occuring and
  % focus only on the effects that scheduling might have on the system state. 
  % To achieve this, a list of random preemption actions is created, which are 
  % then executed and each executed action updates the system state.
  preemption_point(partner: Thread_pointer, allow_timeout: bool, 
                   state_old: System_state): System_state =
    LET 
      % Increment the seed and use it to get a random list of preemption 
      % actions.
      state_old = state_old WITH [seed := state_old`seed + 1],
      actions = preemption_action(state_old`seed)
   IN
      % Execute the preemption actions.
      preemption_point_actions(actions,partner,allow_timeout,state_old)
    
  % Abort the IPC send operation.
  abort_send(partner: Thread_pointer, state_old: System_state): System_state =
    LET
      % Unset the bits that indicate that the sender was busy in IPC.
      bits = TS_empty WITH [thread_send_in_progress := true,
                            thread_polling := true,
                            thread_ipc_in_progress := true,
                            thread_transfer_in_progress := true],
      state_temp = state_del_dirty(state_old`this,bits,state_old),

      state_temp = preemption_point(partner,false,state_temp),

      state_temp =	
        % Check if the sender needs to be dequeued from the sender list.
        IF in_sender_list(partner,state_temp) THEN
          sender_dequeue(partner,state_temp)
        ELSE
          state_temp
        ENDIF,

      state_temp = preemption_point(partner,true,state_temp)    
    IN
      % Remove the thread lock on the receiver.
      clear_dirty(partner,state_temp) WITH [error := true]

  % Handle the finishing part of the do_send_wait() function.
  do_send_wait_finish(partner: Thread_pointer,
                      state_old: System_state): System_state =
    LET
      state_temp = 
        % If a timeout has hit at this point, it can be safely ignored as the
        % handshake has already taken place. However, make sure that the
        % <thread_ipc_in_progress> bit of the <this> thread is set once again 
        % as the timeout will have unset it.
        IF state_old`timeout THEN
        LET
          bits_add = TS_empty WITH [thread_ipc_in_progress := true]
        IN
          state_add_dirty(state_old`this,bits_add,state_old)
        ELSE
          state_old
        ENDIF,
    
      % Add preemption points and dequeue the sender from the sender list.
      state_temp = preemption_point(partner,false,state_temp),
      state_temp = sender_dequeue(partner,state_temp),
      state_temp = preemption_point(partner,false,state_temp),

      % Remove the polling bit as the handshake has ended.
      bits_del = TS_empty WITH [thread_polling := true],
      state_temp = state_del_dirty(state_temp`this,bits_del,state_temp)
    IN
      % Check if IPC was cancelled.
      IF state_temp`threads(state_temp`this)`state`thread_cancel THEN
        % Check if the sender- and receiver are still engaged in IPC, if so the
        % state of the receiver needs to be changed.
        IF NOT in_ipc(state_temp`this,partner,state_temp) THEN
          abort_send(partner,state_temp)
        ELSE
        LET 
          mask = TS_full WITH [thread_ipc_in_progress := false],
          bits = TS_empty WITH [thread_cancel := true, thread_ready := true],
          state_temp = state_change_dirty(partner,mask,bits,state_temp)
        IN  
          abort_send(partner,state_temp)   
        ENDIF
      % Abort the send if the sender is not in IPC with the receiver or the
      % receiver has been killed.
      ELSIF NOT in_ipc(state_temp`this,partner,state_temp) OR 
            state_temp`threads(partner)`state`thread_dead 
      THEN
        abort_send(partner,state_temp)
      ELSE
        % Assert that the sender is not engaged in IPC with the receiver and
        % that the handshake bit is not set.
        state_temp WITH [(assertions_held) := 
                     state_temp`assertions_held AND
        state_temp`threads(state_temp`this)`state`thread_ipc_in_progress AND NOT
                     in_sender_list(partner,state_temp),
                     error := false] 
      ENDIF
 
  % Handle the waiting loop part of the do_send_wait() function, which waits for
  % the receiver to become ready for the sender.
  do_send_wait_loop(partner: Thread_pointer,
                    state_old: System_state): System_state =
    LET
      % Remove the thread lock on the partner as we have to give it a chance in
      % the preemption point to get in the state in which it is ready to receive
      % a message from the <this> thread. Afterwards acquire the thread lock
      % again.
      state_temp = clear_dirty(partner,state_old),
      state_temp = preemption_point(partner,true,state_temp),
      state_temp = lock_dirty(partner,state_temp)
    IN
    % Check if IPC was cancelled or the handshake bit was set, if so finish the
    % do_send_wait() call.
    IF state_temp`threads(state_temp`this)`state`thread_cancel OR 
       state_temp`threads(state_temp`this)`state`thread_transfer_in_progress
    THEN
      do_send_wait_finish(partner,state_temp)
    ELSE
    LET
      % Assert that the sender is not engaged in IPC with the receiver and that
      % the handshake bit is not set.
      state_temp = state_temp WITH [(assertions_held) := 
                       state_temp`assertions_held AND NOT 
                       in_ipc(state_temp`this,partner,state_temp) AND NOT 
         state_temp`threads(state_temp`this)`state`thread_transfer_in_progress]
    IN
      % Check if a timeout has occured, if so abort the sending.
      IF state_temp`timeout THEN
      LET
        % Assert that the sender is not engaged in IPC with the receiver and
        % that the handshake bit is not set.
        state_temp = state_temp WITH [(assertions_held) := 
               state_temp`assertions_held AND NOT 
               state_temp`threads(state_temp`this)`state`thread_ipc_in_progress]
      IN
        abort_send(partner,state_temp)
      % Abort the send if the partner was killed.
      ELSIF state_temp`threads(partner)`state`thread_dead THEN
        abort_send(partner,state_temp)
      ELSE
      LET
        % Assert that the sender is still engaged in the handshake with the
        % receiver.
        state_temp = state_temp WITH [(assertions_held) := 
              state_temp`assertions_held AND 
              state_temp`threads(state_temp`this)`state`thread_ipc_in_progress],

        % Set the polling state
        bits = TS_empty WITH [thread_polling := true], 
        state_temp = state_add_dirty(state_temp`this,bits,state_temp)
      IN
        do_send_wait_finish(partner,state_temp)
      ENDIF
    ENDIF 
  
  % Wait for the receiver to become ready for the sender.
  do_send_wait(partner: Thread_pointer,state_old: System_state): System_state =
    LET
      % Assert that the no other thread but the zero- or <this> thread hold the
      % thread lock on the receiver.
      state_temp = state_old WITH [(assertions_held) :=
                      state_old`assertions_held AND 
                     (state_old`threads(partner)`thread_lock = Zero_thread OR
                      state_old`threads(partner)`thread_lock = state_old`this)],
 
      % Create a thread state which bits to add to the <this> thread.
      add_bits = TS_empty WITH [thread_polling := true,
                                thread_send_in_progress := true,
                                thread_ipc_in_progress := true],
      state_temp = state_add_dirty(state_temp`this,add_bits,state_temp),
      
      % Lock the partner before continuing.
      state_temp = lock_dirty(partner,state_temp),

      % Add a preemption point
      state_temp = preemption_point(partner,false,state_temp) 
    IN
      % Check if IPC was cancelled.
      IF state_temp`threads(state_temp`this)`state`thread_cancel THEN
      LET
        % Remove the thread lock on the receiver.
        state_temp = clear_dirty(partner,state_temp)
      IN
        % Delete the bits from the <this> thread's state.
        state_del_dirty(state_temp`this,TS_ipc_end_mask,
                        state_temp) WITH [error := true]
      ELSE
        % Return if the receiver is ready to receive a message from the <this>
        % thread. If not enter the waiting loop.
        IF sender_ok(state_temp`this,partner,state_temp) THEN
          state_temp WITH [error := false]
        ELSE
        LET
          % Enqueue the sender in the sender list of the receiver.
          state_temp = sender_enqueue(partner,state_temp),

           % Give the receiver time to become ready for the sender.
          state_temp = preemption_point(partner,true,state_temp)
        IN 
          % Abort the sending if a timeout occured.
          IF state_temp`timeout THEN
            abort_send(partner,state_temp)        
          ELSE
            do_send_wait_loop(partner,state_temp)
          ENDIF
        ENDIF
      ENDIF

  % Try to have the sender and receiver agree upon engaging in IPC.
  try_handshake_receiver(partner: Thread_pointer, 
                         state_old: System_state): System_state =
    % Immediately return with an error if the partner is the Zero- or Nil thread
    % or the partner is an invalid thread.
    IF partner = Zero_thread OR
       state_old`threads(partner)`state`thread_invalid OR 
       partner = Nil_thread
    THEN
      state_old WITH [error := true]
    ELSE
    LET 
      % Set the <handshake_attempted> field to true because the receiver is 
      % valid.
      state_temp = state_old WITH [handshake_attempted := true] 
    IN
      LET state_temp = 
        % If a thread lock is already held on the receiver, acquire the thread 
        % lock (after a preemption point has been set).
        IF NOT state_temp`threads(partner)`thread_lock = Zero_thread THEN
        LET state_temp = preemption_point(partner,false,state_temp) IN
          lock_dirty(partner,state_temp)
        ELSE
          state_temp
        ENDIF
      IN
        % Check if the result of the preemption point was that IPC has been 
        % canceled, if so remove the thread lock and return an error.
        IF state_temp`threads(state_temp`this)`state`thread_cancel THEN
          clear_dirty(partner,state_temp) WITH [error := true]
        ELSE
          % IPC has not been cancelled, now check if the receiver is ready for
          % the sender. If not, wait for the receiver to become ready.
          IF NOT sender_ok(state_temp`this,partner,state_temp) THEN
          LET
            state_temp = do_send_wait(partner,state_temp)
          IN
            % Assert that the sender does not hold a thread lock on the receiver
            % when an error occured.
            state_temp WITH [(assertions_held) :=
               state_temp`assertions_held AND 
              (NOT state_temp`error OR 
               NOT state_temp`threads(partner)`thread_lock = state_temp`this)]
          ELSE
            state_temp WITH [error := false]
          ENDIF
        ENDIF
    ENDIF 

  % Handle the send part in the do_ipc() function.
  do_ipc_send_part(partner: Thread_pointer, have_receive: bool, 
                   state_old: System_state): System_state =
    % Start with a handshake and immediately return if an error occured.
    LET state_temp = try_handshake_receiver(partner,state_old) IN
    IF state_temp`error THEN
      state_temp   
    ELSE 
    LET 
      % Assert that after a successful handshake the sender thread is no 
      % longer polling.
      state_temp = state_temp WITH [(assertions_held) := 
                     state_temp`assertions_held AND 
                 NOT state_temp`threads(state_temp`this)`state`thread_polling],
    
      % Initialize IPC.
      state_temp = ipc_init(state_temp`this,partner,state_temp),        
      
      % Transfer the IPC message.
      state_temp = transfer_msg(partner,state_temp),
       
      % Assert that the thread lock on the receiver is held by the <this> thread
      % or that there is no thread lock.
      state_temp = state_temp WITH [(assertions_held) := 
                       state_temp`assertions_held AND 
                      (state_temp`threads(partner)`thread_lock = Zero_thread OR 
                       state_temp`threads(partner)`thread_lock = state_temp`this)]
    IN
      % Check if an error occured or that there is no receive part.
      IF state_temp`error OR NOT have_receive THEN
      LET 
        state_new = 
          % If the partner is still engaged in IPC, wake him up.
          IF in_ipc(state_temp`this,partner,state_temp) THEN
            wake_receiver(partner,state_temp) 
          ELSE
            state_temp
          ENDIF,
        % Remove the thread lock on the partner, adjust the state of the <this>
        % thread and return the error state of the message transfer.
        state_new = clear_dirty(partner,state_new)
      IN
        state_del(state_new`this,TS_ipc_end_mask,
                  state_new) WITH [error := state_temp`error]
      ELSE
      % Remove the thread lock on the partner.
      LET state_temp = clear_dirty_dont_switch(partner,state_temp) IN
        % Return an error if the partner is not engaged in IPC with us. This is
        % because we only arrive here if no error occured during message 
        % transfer and there is a receive part, which means that the partner
        % has yet to send a message to us and should therefore still be engaged
        % in IPC with us. If the partner is still in IPC with us, wake him up 
        % to enable him to send his message.
        IF NOT in_ipc(state_temp`this,partner,state_temp) THEN           
          state_del(state_temp`this,TS_ipc_end_mask,
                    state_temp) WITH [error := true]
        ELSE
          wake_receiver(partner,state_temp)
        ENDIF
      ENDIF
    ENDIF
 
  % Handle the receive part in the do_ipc() function.
  do_ipc_receive_part(sender: Thread_pointer, have_receive: bool,
                      state_old: System_state): System_state =
    state_old WITH [(assertions_held) := 
                     state_old`assertions_held AND
                     have_receive]
     
  % Handle IPC between the sender- and receiver.
  do_ipc(have_send: bool, partner: Thread_pointer, have_receive: bool,
         sender: Thread_pointer, state_old: System_state): System_state =    
    LET state_temp =
      % Check if there is a send part.
      IF have_send THEN
        do_ipc_send_part(partner,have_receive,state_old)
      ELSE
        state_old
      ENDIF
    IN
      % If no error occured in the send part and there is a receive part, 
      % enter the receive part and otherwise return.
      IF have_receive AND NOT state_temp`error THEN
        do_ipc_receive_part(sender,have_receive,state_temp)
      ELSE
        state_temp
      ENDIF

  % Execute an IPC call.
  %
  % NOTE: we tried to make as few assumptions as possible on the system state
  % but there are some fields that have to be initialized before engaging in 
  % IPC. Naturally, before an IPC call is made there has not been any error,
  % timeout, handshake attempted. Furthermore initially the assertions all hold.
  sys_ipc(have_send: bool, partner: Thread_pointer,
          have_receive: bool, sender: Thread_pointer,
          state_old: System_state): System_state = 
    LET state_temp = state_old WITH [(error)   := false,
                                     (timeout) := false,
                                     (handshake_attempted)  := false,
                                     (assertions_held)      := true,
                                     (receiver_initialized) := false]
    IN
      % Make sure there is either a send- or receive part, if so start IPC.
      IF have_send OR have_receive THEN
        do_ipc(have_send,partner,have_receive,sender,state_temp)
      ELSE
        state_temp WITH [(error) := true]
      ENDIF

END fiasco_functions
\end{lstlisting}