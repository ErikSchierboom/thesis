\chapter*{Preface}
To me, computers are fascinating machines in all their aspects. I have always favored software over hardware though, writing my own software was what got me into studying computer science after all. Even though computers play a vital role in society nowadays, one must not forget that computer science is a relatively young field of research\footnote{Especially when compared to other exact sciences such as mathematics and physics, which predate it by many centuries.}. It is therefore not surprising that there are still many problems with computers, although progress is being made rapidly (which is another interesting aspect of computer science). As hardware leaps forward\footnote{Processors are an obvious example, as Moore's law has successfully been applied to them for the last 30 years.}, the general impression is that the software lags behind. This impression stems from the fact that most problems with computers are due to the software running on them.\emptyline

Lately, one of the most interesting developments in writing software, at least to me, has been the shift in focus from performance to dependability\footnote{Which loosely translates to software behaving like one expects it to.}. This shift can, for a large part, be attributed to the changing needs of people. Where one was previously happy that a computer could execute a task, people are now so accustomed to computers that they expect that task to be executed dependably. Unfortunately, if one would ask a random person if he or she has had any bad experiences with the dependability of software, chances are very high that most would answer with a resounding "yes". The prime example of software not being dependable is when it, or the whole system, crashes.\emptyline

Almost all software on business or private computers runs on top of (and thus depends on) an operating system, the most well known probably being Microsoft Windows, Unix/Linux and Mac OS. To me, operating systems are the most interesting pieces of software available because of their vital role and the very diverse tasks they execute. When one wants to create dependable software (which relies on the operating system), it makes sense to create a dependable operating system. In an operating system, the most vital part (or "heart") is the kernel, which dependability is therefore absolutely critical.\emptyline

In line with the desire to create dependable software, verification of software is a rapidly expanding field of computer science. What can be better to its dependability than to formally verify that a piece of software does what it is supposed to do? In our thesis, we will combine both our interest in operating system kernels and software verification. Our research shows how one can verify (a part of) a real kernel. This is particularly useful as kernels are among the most likely candidates for verification because of their vital role. We hope that our thesis will kindle your interest in verification of software, as we expect it to become an integral part of software development in the, hopefully near, future.